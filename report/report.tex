\documentclass[11pt, a4paper]{article}
\usepackage[utf8]{inputenc}
\usepackage[left=2.35cm, right=3.35cm, top=3.35cm, bottom=3.0cm]{geometry}
\usepackage{amsmath, amssymb, amsthm}
\usepackage[english]{babel}
\usepackage{graphicx}
\usepackage[font={small,it}]{caption}
\graphicspath{ {figures/} }
\usepackage{url}
\usepackage{appendix}
\usepackage{float}
\usepackage{multirow}
\usepackage[bottom]{footmisc}
\usepackage{titling}
\usepackage{subcaption}
\usepackage{wrapfig}
\usepackage[numbered,autolinebreaks,useliterate]{mcode}
\begin{document}

\include{title}
\tableofcontents
\newpage

\section*{Context}

The analysis presented in this report was produced for the class of
``Support Vector Machines: methods and applications'' at KU Leuven
(Spring 2016). The goal is to display understanding of the principles
behind support vector machines and of how to work out good solutions
using these techniques. This third report focuses on unsupervised
learning (kernel PCA) using Least-Squares SVM (LS-SVM). The
implementation was done using the MatLab environment (v2015a) and the
libraries for LS-SVM developed at KU Leuven
\footnote{http://www.esat.kuleuven.be/sista/lssvmlab/}.

\section{Kernel Principal Component Analysis}

\section{Handwritten Digit Denoising}

\section{Spectral Clustering}

\section{Fixed-size LS-SVM}

\section{Applications}

\subsection{Handwritten Digit Denoising}

\subsection{Shuttle (statlog)}

\subsection{California}

%\bibliographystyle{ieeetr} \bibliography{bib-db}
\end{document}
